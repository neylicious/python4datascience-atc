
\documentclass{bredelebeamer}


% suppress navigation bar
\beamertemplatenavigationsymbolsempty

\mode<presentation>
{
  %\usetheme{bunsen}
  \setbeamercovered{transparent}
  \setbeamertemplate{items}[circle]
}

\beamertemplatenavigationsymbolsempty
\usepackage{color}
\definecolor{uipoppy}{RGB}{225, 64, 5}
\definecolor{uipaleblue}{RGB}{96,123,139}
\definecolor{uiblack}{RGB}{0, 0, 0}

% caption styling
\DeclareCaptionFont{uiblack}{\color{uiblack}}
\DeclareCaptionFont{uipoppy}{\color{uipoppy}}
\captionsetup{labelfont={uipoppy},textfont=uiblack}
\include{macros}
%%%%%%%%%%%%%%%%%%%%%%%%%%%%%%%%%%%%%%%%%%%%%%%%

\title[PhD]{\textbf{Intro to Python for Data Science \\ Arusha Tech}}
% Titre du diaporama

\subtitle{\textbf{} }
% Sous-titre optionnel

\author{ \href{https://sambaiga.github.io/ }{\href{https://sambaiga.github.io/}{Anthony FAUSTINE} }}





\date{August 2017}
% Optionnel. La date, généralement celle du jour de la conférence

\subject{Sujet de votre diaporama}
% C'est utilisé dans les métadonnes du PDF



\logo{
	%\includegraphics[scale=0.1]{images/logo.pdf}
}
\hypersetup{
	pdfauthor = {Anthony Faustine: sambaiga@gmail.com},
	pdfsubject = {},
	pdfkeywords = {},
	pdfmoddate= {},
	pdfcreator = {}
}




%%%%%%%%%%%%%%%%%%%%%%%%%%%%%%%%%%%%%%%%%%%%%%%%%%%%%%%%%%%%%%%%%%%%%
\begin{document}


\begin{frame}
	\titlepage
\end{frame}




\AtBeginSection[] { \begin{frame} %
		<beamer> \frametitle{Outline} \setcounter{tocdepth}{2} \tableofcontents[currentsection, sectionstyle=show/shaded,hideallsubsections ] \end{frame} }



%++++++++++++++++++++++++++++++++++++++++++++++++++++

\section{Introduction}

\begin{frame}{Learning goal}
\begin{itemize}
	\item Understand python programming language and different python libraries for data science.
	\item Explore Python language fundamentals, including basic syntax, variables, control flow, data structure and functions.
	\item Build Numpy arrays, and perform basic and some linear algebra calculations.
	\item Create and customize plots  using matplotlib.
\end{itemize}
\end{frame}
\begin{frame}{Presenter Bio}
\begin{itemize}
\item PhD student at Nelson Mandela African Institution of Science and Technology, 
\item \textbf{Research:}  Applied machine learning and signal processing for computational sustainability.
\begin{itemize}
	\item Develop probabilistic-deep learning algorithm (Hybrid HMM-DNN) for energy dis-aggregation problem.
\end{itemize}	
\item Co-founder \href{https://pythontz.github.io/}{Pythontz}
\item Assistant Lecturer (UDOM), Researcher (Vicres, \href{https://hakikidawa.github.io/}{Hakikidawa}).
\end{itemize}
\end{frame}


\begin{frame}{Pythontz}
\begin{figure}
\centering
\includegraphics[scale=0.40]{../image/pythontz.jpg}
\end{figure}
\end{frame}


\begin{frame}{Pythontz}
\emph{About Pythontz}
\begin{itemize}
	\item A postive peer learning community for interested Python users in Tanzania.
\end{itemize}
\textbf{Vision}
\begin{itemize}
	\item To create a vibrant and diverse python community in Tanzania.
\end{itemize}
\textbf{Mission}
\begin{itemize}
	\item To foster the application of python programming across industries, learning centers, schools and community in Tanzania.
\end{itemize}
\end{frame}
\section{Python}

\begin{frame}\frametitle{Introduction}
\framesubtitle{What is Python ?}
A very popular general-purpose programming language.
\begin{itemize}
\item Open source general-purpose language
\item Dynamically semantics (rather than statically typed like Java or C/C++)
\item Interpreted (rather than compiled like Java or C/C++)
\item Object Oriented, 
\end{itemize}
\end{frame}	

\begin{frame}\frametitle{What can you use Python for?}

\begin{itemize}
\item Web development (\href{https://www.djangoproject.com/}{Django})
\item Web Scraping (\href{https://www.djangoproject.com/}{Beautiful Soup})
\item Scripting Language.
\item Scientific programming and Numeric Computing.
\item Automation and Embedded Sytstem.
\item Desktop GUIs and 3D modelling. 
\end{itemize}
\end{frame}	
	
\begin{frame}\frametitle{But Why Python ?}
\begin{multicols}{2}
\begin{figure}[h]
		\includegraphics[scale=0.30]{../image/whypython.png}
		\label{fig:result1}
		\caption{Jake VanderPlas PyCon 2017}
\end{figure}
\columnbreak
\begin{itemize}
	\item Python is a “teaching language”
	\item ....created to “bridge the gap between the shell and C
	\item  “never intended. . . to be the primary language for programmers.”
\end{itemize}

\end{multicols}	
\end{frame}	

\begin{frame}{Why is Python such an effective tool in
science?}
\begin{enumerate}[<+>]
	\item Interoperability with Other Languages: You can use it in the shell on microtasks, or interactively, or in scripts, or build enterprise software with GUIs.
	\item “Batteries Included” + Third-Party Modules: Python has built-in libraries and third-party liabraies for nearly everything.
	\item Simplicity \& Dynamic Nature: You can run your Python code on any architecture.	
	\item Open ethos well-fit to science: Easy to reproduce results with python	
	\item Python is the future of Machine Learning and AI.
\end{enumerate}
\begin{center}	
\textit{Jake VanderPlas PyCon 2017}
\end{center}
\end{frame}

\begin{frame}{Why is Python such an effective tool for Data Science}
\begin{enumerate}[<+>]
	\item Very rich scientific computing libraries
    \item All DS tasks can be performed with Python:
    \begin{itemize} 
  \item accessing, collecting, cleaning, analysing, visualising data 
   \item modelling, evaluating models, integrating in prod, scaling
  \end{itemize}
\end{enumerate}
\begin{center}	
\textit{http://slides.com/utstikkar/introtopython-pythonproglanguage\#/3}
\end{center}
\end{frame}


\begin{frame}{PYTHON 2 VS. PYTHON 3}
\begin{itemize}
	\item 2 major versions of Python in widespread use:
Python 2.x and Python 3.x
\item Some features in Python 3 are not backward compatible with Python 2
\item Some Python 2 libraries have not been updated to work with Python 3
\item Bottom-line: there is no wrong choice, as long as all the libraries you need are supported by the version you choose.
\item In this workshop: Python3
\end{itemize}	

\end{frame}



\begin{frame}{Resource to learn Python}

{\LARGE
\href{https://simpleprogrammer.com/2017/02/15/get-started-learning-python/}{10 Resources to Get Started Learning Python}
}
\end{frame}

	
\section{Data Science}
\begin{frame}{What is Data science}
\texttt{The future belongs to the companies and people that turn data into products.
By Mike Loukides June 2, 2010}\\[0.5cm]
\pause
\emph{Data science:}  deals with analyzing and manipulating data to derive insights and build data products.
\begin{itemize}
	\item The end goal of DS $\Rightarrow$ data product:\\[0.25cm] \pause
	 \alert{Data product}: any tool created with the help of data to make a more informed decision.
\end{itemize}	
\end{frame}

\begin{frame}{Data science vs Machine learning}

\emph{Machine learning:}  a set of algorithms that learn from data in order to make predictions or inference.
\begin{itemize}
	\item Data Science is the real-world application of machine learning, with the goal of creating data products.
\end{itemize}	
\end{frame}
\section{Python Packages for Data Science}
\begin{frame}{Python’s Scientific Stack}
\begin{figure}[h]
		\includegraphics[scale=0.40]{../image/pythonstack.jpg}
\end{figure}
\end{frame}
\begin{frame}{Jupyter}
\emph{Jupyter}: Open-source web application for interactive and exploratory computing.
\begin{itemize}
\item Allows to create and share documents that contain live code, equations, visualizations and explanatory text.
\end{itemize}
\begin{multicols}{2}
\begin{figure}[h]
		\includegraphics[scale=0.30]{../image/jupyter.png}
		\label{fig:result1}
\end{figure}
\begin{itemize}
	\item It is a platform for Data Science at scale.
	\item Covers all the life-cycle of scientific ideas:ideas to publications.
	\item \href{https://try.jupyter.org/}{Demo}
\end{itemize}

\end{multicols}	
\end{frame}	


\begin{frame}{Numpy and Sci-py}
\begin{multicols}{2}
\href{http://www.numpy.org/}{\emph{Numpy}}: the fundamental Python package for scientific computing.
\begin{figure}[h]
		\includegraphics[scale=0.30]{../image/numpy.jpg}
		\label{fig:result1}
\end{figure}

\begin{itemize}
	\item Provide high-performance vector, matrix and higher-dimensional data structures.
	\item Offers Matlab-ish capabilities within Python.
\end{itemize}
\href{https://www.scipy.org/}{\emph{Sci-py}}: Collections of high level mathematical operations
\begin{figure}[h]
		\includegraphics[scale=0.30]{../image/scipy.jpg}
		\label{fig:result1}
\end{figure}
\begin{itemize}
	\item linear algebra.
	\item Optimization
	\item Integration etc.
\end{itemize}
\end{multicols}	
\end{frame}	



\begin{frame}{statsmodels}
statsmodels: statistical modelling toolbox 

\end{frame}	


\begin{frame}{Matplotlib}
\href{https://matplotlib.org/}{Matplotlib} is an excellent 2D and 3D graphics library for generating scientific figures.
 \begin{itemize}
 	\item It provides both a very quick way to visualize data from Python and publication-quality figures in many formats.
 \end{itemize}
 \begin{figure}[h]
		\includegraphics[scale=0.40]{../image/matplotlib.png}
		\label{fig:result1}
\end{figure}
Other data visualization packages: \href{https://seaborn.pydata.org/}{Seaborn} and \href{http://bokeh.pydata.org/en/latest/}{Bokeh}.
\end{frame}

\begin{frame}{Other Python Library for Visualization}
 \begin{figure}{}
\includegraphics[scale=0.45]{../image/searborn.png}
\end{figure}
\end{frame}

\begin{frame}{Pandas}
\href{http://pandas.pydata.org/}{Panda}: a python package providing fast, flexible, and expressive data structures for data analysis.
\begin{itemize}
	\item A fundamental high-level building block for doing practical, real world data analysis in Python.
	\item Designed to work with relational or labeled data or both.
\end{itemize}	
\begin{figure}[h]
		\includegraphics[scale=0.30]{../image/pandas.png}
\end{figure}
\end{frame}


\begin{frame}{Scikit-Learn for ML}
\href{http://scikit-learn.org/}{Scikit-Learn (sklearn)} is Python's premier general-purpose machine learning library. 
\centering
\includegraphics[scale=0.30]{../image/sklearn.png}
\end{frame}

\begin{frame}{Python ML and AI libraries}
\begin{multicols}{4}
\href{https://www.tensorflow.org/}{Tensorflow}
\begin{figure}[h]
\includegraphics[scale=0.25]{../image/tf.jpg}
\end{figure} 
\href{http://deeplearning.net/software/theano/}{Theano}
\begin{figure}[h]
\includegraphics[scale=0.25]{../image/theano.jpg}
\end{figure}
\href{http://pytorch.org/}{Pytorch}
\begin{figure}[h]
\includegraphics[scale=0.2]{../image/pytorch.png}
\end{figure}
\href{https://keras.io/}{Keras}
\begin{figure}[h]
\includegraphics[scale=0.25]{../image/keras.png}
\end{figure}
\href{http://edwardlib.org/}{Edward}
\begin{figure}[h]
\includegraphics[scale=0.1]{../image/edward.png}
\end{figure}
\href{http://pymc-devs.github.io/pymc3/}{PyMC3}
\begin{figure}[h]
\includegraphics[scale=0.25]{../image/pymc3.png}
\end{figure}
\href{http://www.nltk.org/}{NLTK}
\begin{figure}[h]
\includegraphics[scale=0.2]{../image/nltk.jpg}
\end{figure}
\end{multicols}
\end{frame}


\begin{frame}{Data Science Platform}
\href{https://www.kaggle.com/}{Kaggle}: helps you learn, work, and play.
\begin{figure}[h]
		\includegraphics[scale=0.25]{../image/kaggle1.png}
\end{figure}
Data set:
\begin{itemize}
	\item \href{http://academictorrents.com/}{Academic Torrents}
	\item \href{https://archive.ics.uci.edu/ml/datasets.html}{UCI Machine learning repository}
\end{itemize}	
\end{frame}
%-------------------------------------------------------------
\begin{frame}
\centering
\emph{THANK YOU}
\end{frame}

\begin{frame}{Practical Session}

\centering
{\LARGE
    Practical Session 
}
\end{frame}

\end{document}

